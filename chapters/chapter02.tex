The first part of this chapter will define a toy model of reality.
A key difference will be that the amoeba will release a single cAMP with no initial velocity for each spawning moment.
The second important difference is that the release of cAMP will be regulated by a Poisson process, thus ignoring the fact that in reality the amoebas should respond to each trace of cAMP with the release of a chemical cry of more chemotaxin.
The second part of the chapter is concerned with the definition of an alternative version of this first design, which can be handy for certain mathematical operations.
Recall that the space is throughout the whole master thesis $\R^d$ with $d\in\{2,3\}$, thus not the phase space.

\section{Deterministic parenting toy model}

In words we can describe this design in this way:
\begin{enumerate}
\item There is a fixed number $N$ of amoebas, each subjected to its own Brownian motion with constant diffusion coeffiecient $\sigma^\amba$
\item Each amoeba $\amba^p$ attracted to the cAMP molecules with a potential $\potential$ which has properties [[PROPERTIES MODELLED ON THE GAUSSIAN]]
\item Each amoeba $\amba^p$ spawns cAMP molecules $\camp^{p,g}$ (p=parent, g=generation) with a Poisson process $\hmactive^p$ of parameter $\lambda$ indipendent of $p$
\item Each cAMP molecule is born at the position of its parent and is subjected from that time on to its own Brownian motion with constant diffusion coefficient $\sigma^\camp$
\item All the Brownian motions for each particle are indipendent from each others
\end{enumerate}

Let us work more precisely.

Let $N\in\N$ be the number of amoebas, and fix $N$ indipendent Poisson processes $(\hmactive^p)_{i=1}^N$ of parameter $\lambda$.
Define for each parent $p \in \{1,\ldots,N\}$ and each generation $g \in \N$, the stopping time

\begin{equation}
\btime^{p,g} := \inf \{ t : \hmactive^p_t \geq g \}
\end{equation}

\ie the \emph{spawning} or \emph{birth time} of the $\camp^{p,g}$ cAMP molecule.

[[HERE ILLUSTRATION OF WHAT IS HAPPENING TIMEWISE]]

Consider a countable collection of indipendent Brownian motions 

\begin{equation}
\{B^{A,p}, B^{C,p,g}\}_{1\leq p\leq N,g\in\N} =:\{B^p, B^{p,g}\}_{p,g}
\end{equation}

\ie $N$ of them for the amoebas and a countable amount for the cAMP molecules.
Now we need an important passage to accurately define the Brownian movement of the cAMP particles.
In fact each of these moves only after spawning, and thus such movement must be calibrated for starting after this moment, and not before.
For each $1\leq p \leq N,g \in \N$ consider the process

\begin{equation}
B_t^{p,g,(\btime^{p,g})} := B_{\btime^{p,g}+t}^{p,g} - B_{\btime^{p,g}}^{p,g}
\end{equation}

Such process is a Brownian motion because of the strong Markov property (\citep[Theorem 2.20]{le-gall}), but this is shifted of a time $T^{p,g}$.
Wat we need is a "quasi-Brownian motion", which is $0$ till the time $T^{p,g}$ and afterward starts without shifts forward, thus this object:

\begin{figure}[h]
\begin{tikzpicture}
\end{tikzpicture}
\caption{[[ Add HERE TIKZ PICTURE SHOWING IN DIFFERENT COLORS THE VARIOUS PROCESSES INVOLVED]]}
\end{figure}

\begin{equation}\label{quasi-bm-camp}
\quasibm_t^{p,g} := "B_{t-\btime^{p,g}}^{p,g,(\btime^{p,g})}\1_{t-\btime^{p,g}\geq 0}"
\left\{
\begin{aligned}
& 0 
& t<T^{p,g}
\\
& B_{t-\btime^{p,g}}^{p,g,(\btime^{p,g})} = B_{t}^{p,g} - B_{\btime^{p,g}}^{p,g}
& t \geq \btime^{p,g}
\end{aligned}
\right.
\end{equation}

With these indipendent processes we can now finally write down the model as a set of stochastic differential equations:

\begin{align*}
\left\{
\begin{aligned}
A_t^p 
&= A_0^p 
+ \int_0^t \frac{1}{N} \sum _{p = 1}^{N} \sum_{g = 1}^{\hmactive_s^p} \drift(A_s^p - C_s^{p,g}) d s
+ 
\overbrace{\int_0^t \sigma_A d B_s^p}^{\sigma^A B_t^p}
&
\\
C_t^{p,g} 
&\qquad\text{does not exist} 
&if \quad t < \btime^{p,g}
\\
C_t^{p,g}
&= A_{\btime^{p,g}}^p + \underbrace{\int_0^t \sigma_C Q_s^{p,g}}_{\sigma_C \quasibm_t^{p,g}} 
&if \quad t \geq \btime^{p,g}
\end{aligned}
\right.
\end{align*}\label{model:deterministic-parenting}

One can immediately notice that instead, not even the dimension of the system \ref{model:deterministic-parenting} is deterministic: the particles $C_t^{p,g}$ are spawning at the stochastic time $\btime^{p,g}$, increasing at each of these instants the number of particles involved (the system is piecewise defined in time).
This, together with the fact that the Poisson processes $\hmactive^p$ cannot be represented with a SDE and that the $Q_t^{p,g}$ are not proper Brownian motions, makes this system formally not a system of SDEs, even though the equations involved are indeed differential and stochastic.

One could argue that a possible solution to render at least the dimension not stochastic could be done like this: let all the particles exist since the instant $0$, and simply consider a subset of them, the "active" ones, for the attraction of the amoebas.
This is indeed a good method, but, as we will see, would then create great difficulties later. 
In fact at the spawning moment the particle $C^{p,g}$ must be at the position $A_{\btime^{p,g}}^p$, which means that either it suddenly jumps at such position, making hard to prove the existence and uniqueness of solutions for the system because of the non-easy-to-deal-with kind of discontinuouty in time ([[HERE INSERT THE PROPER REFERENCE TO THE CHAPTER OF EXISTENCE AND UNIQUENESS, THIS HAPPENS WHEN THE PARTICLES JUMP FROM THEIR FIXED POSITION TO THE SPAWNING POSITION]]) or would have to follow the amoeba till the moment of spawning, rendering hard the proper definition and comparison of the relative the mean-field particles ([[HERE INSERT THE PROPER REFERENCE TO THE CHAPTER OF mean-field particles, THIS HAPPENS WHEN THE PARTICLES FOLLOW THEIR PARENTS TILL THE SPAWNING MOMENTS]]).


\subsection{Existence and uniqueness of solutions}

The best chance we have to prove existence and uniqueness of solutions for the defined system is to find a way to see it (or some of its parts) as proper SDEs and use the theorems of such framework to reach the goal.

Recall that an SDE, according to the mathematical definition, is an equation of the kind

\begin{equation}\label{eq:general-sde}
Y_t = Y_0 + \int_0^t b(t, Y_t) \,dt + \int_0^t\sigma(t, Y_t) \,d B_t
\end{equation}

of finite deterministic dimension.
A solution to it is a couple of a continuous stochastic process $Y$ and a brownian motion $B$ fullfilling the equation \ref{eq:general-sde}.

Our system presents a considerable amount of difficulties:
\begin{itemize}
\item An hypothetical vector $\modsol_t := (A_t^1,\ldots, A_t^N, \text{camp molecules } C_t^{p,g} \text{ already spawned at time }t)$ would have a dimension growing to infinite as the time flows.
\item Since the system is piecewise defined in time, at every instant $t$ the same hypothetical vector $\modsol_t$ would have a finite \emph{stochastic} dimension.
\item If we were to write the equations for $\modsol$ in a similar way to the one in $\ref{eq:general-sde}$ we would have a term $b = b(t, X_t, (M_t^p)_p)$ depending also on the Poisson-distributed processes $(M_t^p)_p$ (thus not representable as solution of an SDE), making it "too much stochastical" to be treated as in \ref{eq:general-sde}.
\end{itemize}

To make the dimension finite we can operate in this way:

\begin{equation*}
\exists ! X_t \text{solution to \ref{eq:general-sde} in }\R_+ \Longleftrightarrow \forall \overline{t} \in \R_+ \exists ! . X_t \text{solution to \ref{eq:general-sde} in } [0,\overline{t}] 
\end{equation*}

Let us then fix $\overline{t}\in\R$ and consider the system only on $[0,\overline{t}]$.
In such time each amoeba $A^p$ will have spawned exactly $\hmactive^p_{\overline{t}}$ particles, and thus the maximum dimension the system reaches is in the instant $\overline{t}$ and is given by the total number of particles present.
If we define the process $\hmactive_t := \sum_{p=1}^N \hmactive_t^p$ the number of $d$-dimensional equations is given by:

\begin{equation}
D := \overbrace{N}^{\#\text{amoebas}} 
+ \overbrace{\hmactive_{\overline{t}}}^{\#\text{(cAMP spawned till time }\overline{t})}
\end{equation}

and thus the maximum dimension of the system is $D\cdot d = N + M_{\overline{t}}$, which is finite, but still stochastic.

It is clear that 

\begin{equation}
\forall p. \{\btime^{p,g} \leq s\} = \{ \hmactive_s^p \geq g \}
\end{equation}

and thus choosing a realization of the $\Pois(\lambda)$-process $(\hmactive_s^p)_{0\leq s \leq \overline{t}}$ is equivalent to choose a realization of the $\Gma(g,\lambda)$ birth times $(\btime^{p,g})_{g}$.
We also do notice that the birth times are not dependent on the amoebas and cAMPs molecules movements.
The idea is then to pass from a stochastic to a finite dimension by considering the equation under the regular probability measure $\Prb(\cdot | \bvector)$, where $\bvector$ is defined as

\begin{equation}
\bvector := (\btime^{p,g})_{p,g}
\end{equation}

In this way the dimension of the system can be treated as constant, as if we acted realization-wise for each realization of the birth-times.

More details on the verification of this step, and on why the random measure $\Prb(\cdot | \bvector)$ is regular can be found in the appendix at \ref{app:regular-random-measure}.

This solves also the problem of the drift term:












To reach a deterministic dimension we can fix the times and treat them as deterministic ([[PUT ON APPENDIX A BETTER EXPLANATION FOLLOWING THE DISINTEGRATION OF MEASURES AND PROBABILITY MEASURES FROM OLAV KALLNBERG, theorem 8.5]])





\section{Random (casual) parenting toy model}

Another solution of 


Here define the 2 models involved (the casual and random parenting). 

At the others you work easily afterwards