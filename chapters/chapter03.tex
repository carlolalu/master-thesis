The best chance we have to prove existence and uniqueness of solutions for the defined system is to find a way to see it (or some of its parts) as proper SDEs and use the theorems of such framework to reach the goal.


\section{SDE framework}

An SDE is an expression of the form:

\begin{equation}\label{ch3:eq:general-sde}
\left\{
\begin{aligned}
d X_t &= b(t, X_t) dt + \sigma(t, X_t) d B_t	\\
X_0 &\sim \rho_0
\end{aligned}
\right.
\end{equation}

which is a more conceptual notation to formally mean

\begin{equation}\label{ch3:eq:general-sde-integral-form}
X_t = \rho_0 + \int_0^t b(t, X_t) \,dt + \int_0^t\sigma(t, X_t) \,d B_t
\end{equation}

where $b\in \R^\Iu$, $\sigma \in \R^{\Iu, \Ch}$. A solution is a couple $(X, B)$ such that $B_t\in \R^\Ch$ is Wiener process and $X$ is a \emph{continous} semi-martingale.

\paragraph{cAMP particles cannot jump}\label{ch3:avoid-pitfall-of-jumping-particles} Recall that in \ref{ch2:pitfall} we hinted that a possible solution to avoid a dimension changing with the time (a piecewise defined system) was to let cAMP particles $\camp^{p,g}$ exist since the beginning in a fixed position and then at spawning time jump at position $\amba_{T^{p,g}}^p$ and start moving of Brownian motion.
This is not a feasible way to define the system because then we wouldn't be able to reach an expression which has a continous solution as an SDE: the particles cannot jump!

To ensure existence (strong uniqueness) and uniqueness (path uniqueness) of solutions we need to require conditions on $b, \sigma$. There are two main results:
\begin{itemize}
\item \autocite[Thm 8.3]{le-gall}: The two coefficients are Lipschitz (eventually only locally) in space and continuous:
\begin{equation*}
\left\{
\begin{aligned}
\forall t .( |b(t, x)-b(t,y)| &< L_b |x-y| &\text{ and } |\sigma(t, x)-\sigma(t,y)| &< L_\sigma |x-y|) \\
b &\in \mathcal{C}^0 &\text{ and }\sigma &\in \mathcal{C}^0
\end{aligned}
\right.
\end{equation*}
\item \autocite[Thm 9.4]{baldi}: The two coefficients are Lipschitz in space (as before) and have sublinear growth in space:
\begin{equation*}\label{eq:lipsch-subgrowth}
\left\{
\begin{aligned}
\forall t .(|b(t, x)-b(t,y)| &< L_b |x-y| &\text{ and } |\sigma(t, x)-\sigma(t,y)| &< L_\sigma |x-y|) \\
\forall t |b(t, x)| &< M_b(1 + |x|) &\text{ and } |\sigma(t, x)| &< M_\sigma (1 + |x|)
\end{aligned}
\right.
\end{equation*}
\end{itemize}

Our system presents a considerable amount of difficulties:
\begin{itemize}
\item An hypothetical vector $\modsol_t := (\amba_t^1,\ldots, \amba_t^N, \text{camp molecules } \camp_t^{p,g} \text{ already spawned at time }t)$ would have a dimension growing to infinite as the time flows.
\item Since the system is piecewise defined in time, at every instant $t$ the same hypothetical vector $\modsol_t$ would have a finite \emph{stochastic} dimension.
\item If we were to write the equations for $\modsol$ in a similar way to the one in $\ref{eq:general-sde}$ we would have a term $b = b(t, X_t, (M_t^p)_p)$ depending also on the Poisson-distributed processes $(M_t^p)_p$ (thus not representable as solution of an SDE), making it "too much stochastical" to be treated as in \ref{eq:general-sde}.
\end{itemize}

To make the dimension finite we can operate in this way:

\begin{equation*}
\exists ! X_t \text{solution to \ref{eq:general-sde} in }\R_+ \Longleftrightarrow \forall \overline{t} \in \R_+ \exists ! . X_t \text{solution to \ref{eq:general-sde} in } [0,\overline{t}] 
\end{equation*}

Let us then fix $\overline{t}\in\R$ and consider the system only on $[0,\overline{t}]$.
In such time each amoeba $\amba^p$ will have spawned exactly $\hmactive^p_{\overline{t}}$ particles, and thus the maximum dimension the system reaches is in the instant $\overline{t}$ and is given by the total number of particles present.
Then the number of $d$-dimensional equations is given by:

\begin{equation}
D := \overbrace{N}^{\#\text{amoebas}} 
+ \overbrace{\hmactive_{\overline{t}}}^{\#\text{(cAMP spawned till time }\overline{t})}
\end{equation}

and thus the maximum dimension of the system is $D\cdot d = N + M_{\overline{t}}$, which is finite, but still stochastic.




























%
%
%\subsection{The system as a piecewise deconditioned SDE}
%If we define the process $\hmactive_t := \sum_{p=1}^N \hmactive_t^p$ 
%
%It is clear that 
%
%\begin{equation}
%\forall p. \{\btime^{p,g} \leq s\} = \{ \hmactive_s^p \geq g \}
%\end{equation}
%
%and thus choosing a realization of the $\Pois(\lambda)$-process $(\hmactive_s^p)_{0\leq s \leq \overline{t}}$ is equivalent to choose a realization of the $\Gma(g,\lambda)$ birth times $(\btime^{p,g})_{g}$.
%We also do notice that the birth times are not dependent on the amoebas and cAMPs molecules movements.
%
%
%
%
%
%
%%The idea is then to pass from a stochastic to a finite dimension by considering the equation under the regular probability measure $\Prb(\cdot | \bvector)$, where $\bvector$ is defined as
%%
%%\begin{equation}
%%\bvector := (\btime^{p,g})_{p,g}
%%\end{equation}
%
%In this way $(M_t^p)_p$ can be treated as a deterministic function in time, and the dimension of the system can be treated as constant.
%
%More details on the verification of this reasoning, and on why the random measure $\Prb(\cdot | \bvector)$ is regular, can be found in the appendix at \ref{app:regular-random-measure}.
%
%Then now \ref{model:deterministic-parenting} can be rewritten in a way compatible with the SDE framework.
%
%\begin{equation}
%\modsol_t = \modsol_0 + \int_0^t b(s, \modsol_s, (\hmactive_s^p)_p) \,ds + \int_0^t\sigma \,d B_s
%\end{equation}
%
%where $b(s,x) = b(\cdot, \cdot, (\hmactive_t^p)_p)): \R_+ \times \R^{Dd} \longrightarrow \R^{Dd}$ is defined as
%
%\begin{equation}
%b(s, x, (\hmactive_t^p)_p) :=
%\left\{
%\begin{aligned}
%& \frac{1}{N} \sum _{p = 1}^{N} \sum_{g = 1}^{\hmactive_s^p} \drift(x^p - ^{p,g})
%& if 
%\\
%\camp_t^{p,g} 
%&\qquad\text{does not exist} 
%&if \quad t < \btime^{p,g}
%\\
%\camp_t^{p,g}
%&= \amba_{\btime^{p,g}}^p + \underbrace{\int_0^t \sigma_C Q_s^{p,g}}_{\sigma_C \quasibm_t^{p,g}} 
%&if \quad t \geq \btime^{p,g}
%\end{aligned}
%\right.
%\end{equation}\label{def:b-term-deterministic-parenting}
%
%and $\sigma \in \R^{Dd}\times\R^{Dd}$ is the constant matrix:
%
%\begin{equation}
%\begin{aligned}
%matrix
%\end{aligned}
%\end{equation}
%
%
%
%
%
%
%
%
%
%
%To reach a deterministic dimension we can fix the times and treat them as deterministic ([[PUT ON APPENDIX A BETTER EXPLANATION FOLLOWING THE DISINTEGRATION OF MEASURES AND PROBABILITY MEASURES FROM OLAV KALLNBERG, theorem 8.5]])
%
%
%

