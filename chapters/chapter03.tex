The best chance we have to prove existence and uniqueness of solutions for the defined system is to find a way to see it (or some of its parts) as proper SDEs and use the theorems of such framework to reach the goal.

\subsection{The system as a piecewise deconditioned SDE}

Recall that an SDE, according to the mathematical definition, is an equation of the kind

\begin{equation}\label{eq:general-sde}
Y_t = Y_0 + \int_0^t b(s, Y_s) \,ds + \int_0^t\sigma(s, Y_s) \,d B_s
\end{equation}

of finite deterministic dimension.
A solution to it is a couple of a continuous stochastic process $Y$ and a brownian motion $B$ fullfilling the equation \ref{eq:general-sde}.

Our system presents a considerable amount of difficulties:
\begin{itemize}
\item An hypothetical vector $\modsol_t := (A_t^1,\ldots, A_t^N, \text{camp molecules } C_t^{p,g} \text{ already spawned at time }t)$ would have a dimension growing to infinite as the time flows.
\item Since the system is piecewise defined in time, at every instant $t$ the same hypothetical vector $\modsol_t$ would have a finite \emph{stochastic} dimension.
\item If we were to write the equations for $\modsol$ in a similar way to the one in $\ref{eq:general-sde}$ we would have a term $b = b(t, X_t, (M_t^p)_p)$ depending also on the Poisson-distributed processes $(M_t^p)_p$ (thus not representable as solution of an SDE), making it "too much stochastical" to be treated as in \ref{eq:general-sde}.
\end{itemize}

To make the dimension finite we can operate in this way:

\begin{equation*}
\exists ! X_t \text{solution to \ref{eq:general-sde} in }\R_+ \Longleftrightarrow \forall \overline{t} \in \R_+ \exists ! . X_t \text{solution to \ref{eq:general-sde} in } [0,\overline{t}] 
\end{equation*}

Let us then fix $\overline{t}\in\R$ and consider the system only on $[0,\overline{t}]$.
In such time each amoeba $A^p$ will have spawned exactly $\hmactive^p_{\overline{t}}$ particles, and thus the maximum dimension the system reaches is in the instant $\overline{t}$ and is given by the total number of particles present.
If we define the process $\hmactive_t := \sum_{p=1}^N \hmactive_t^p$ the number of $d$-dimensional equations is given by:

\begin{equation}
D := \overbrace{N}^{\#\text{amoebas}} 
+ \overbrace{\hmactive_{\overline{t}}}^{\#\text{(cAMP spawned till time }\overline{t})}
\end{equation}

and thus the maximum dimension of the system is $D\cdot d = N + M_{\overline{t}}$, which is finite, but still stochastic.

It is clear that 

\begin{equation}
\forall p. \{\btime^{p,g} \leq s\} = \{ \hmactive_s^p \geq g \}
\end{equation}

and thus choosing a realization of the $\Pois(\lambda)$-process $(\hmactive_s^p)_{0\leq s \leq \overline{t}}$ is equivalent to choose a realization of the $\Gma(g,\lambda)$ birth times $(\btime^{p,g})_{g}$.
We also do notice that the birth times are not dependent on the amoebas and cAMPs molecules movements.






The idea is then to pass from a stochastic to a finite dimension by considering the equation under the regular probability measure $\Prb(\cdot | \bvector)$, where $\bvector$ is defined as

\begin{equation}
\bvector := (\btime^{p,g})_{p,g}
\end{equation}

In this way $(M_t^p)_p$ can be treated as a deterministic function in time, and the dimension of the system can be treated as constant.

More details on the verification of this reasoning, and on why the random measure $\Prb(\cdot | \bvector)$ is regular, can be found in the appendix at \ref{app:regular-random-measure}.

Then now \ref{model:deterministic-parenting} can be rewritten in a way compatible with the SDE framework.

\begin{equation}
\modsol_t = \modsol_0 + \int_0^t b(s, \modsol_s, (\hmactive_s^p)_p) \,ds + \int_0^t\sigma \,d B_s
\end{equation}

where $b(s,x) = b(\cdot, \cdot, (\hmactive_t^p)_p)): \R_+ \times \R^{Dd} \longrightarrow \R^{Dd}$ is defined as

\begin{equation}
b(s, x, (\hmactive_t^p)_p) :=
\left\{
\begin{aligned}
& \frac{1}{N} \sum _{p = 1}^{N} \sum_{g = 1}^{\hmactive_s^p} \drift(x^p - ^{p,g})
& if 
\\
C_t^{p,g} 
&\qquad\text{does not exist} 
&if \quad t < \btime^{p,g}
\\
C_t^{p,g}
&= A_{\btime^{p,g}}^p + \underbrace{\int_0^t \sigma_C Q_s^{p,g}}_{\sigma_C \quasibm_t^{p,g}} 
&if \quad t \geq \btime^{p,g}
\end{aligned}
\right.
\end{equation}\label{def:b-term-deterministic-parenting}

and $\sigma \in \R^{Dd}\times\R^{Dd}$ is the constant matrix:

\begin{equation}
\begin{aligned}
matrix
\end{aligned}
\end{equation}










To reach a deterministic dimension we can fix the times and treat them as deterministic ([[PUT ON APPENDIX A BETTER EXPLANATION FOLLOWING THE DISINTEGRATION OF MEASURES AND PROBABILITY MEASURES FROM OLAV KALLNBERG, theorem 8.5]])




