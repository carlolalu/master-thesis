The best chance we have to prove existence and uniqueness of solutions for the defined system is to find a way to see it (or some of its parts) as proper SDEs and use the theorems of such framework to reach the goal.

\section{The SDEs framework and the conditions for existence and uniqueness}

An SDE is an expression of the form:

\begin{equation}\label{ch3:eq:general-sde}
\left\{
\begin{aligned}
d X_t &= \drift(t, X_t) dt + \diffusion(t, X_t) d B_t	\\
X_0 &\sim \rho_0
\end{aligned}
\right.
\end{equation}

which is a more conceptual notation to formally mean

\begin{equation}\label{ch3:eq:general-sde-integral-form}
X_t = \rho_0 + \int_0^t \drift(t, X_t) \,dt + \int_0^t \diffusion(t, X_t) \,d B_t
\end{equation}

where $\drift \in \R^\Iu$, $\diffusion \in \R^{\Iu \times \Ch}$ are given vectorial and matricial functions respectively (the given initial distribution $\rho$ is optional).
They are called \emph{drift} and \emph{volatiliyt} respectively.
We sign such SDE as a $E(\drift,\diffusion)$.
\begin{definition}
A solution for $E(\drift,\diffusion)$ consists of
\begin{itemize}
\item a complete filtered space $(\Omega, \Flt, (\Flt_t)_t, \Prb) $
\item an $\Ch$-dimensional $(\Flt_t)_t$-Brownian motion $B = (B^1, \ldots, B^\Ch)$
\item an $(\Flt_t)_t$-adapted $\Iu$-dimensional process $X_t$ with $\Cspc^0$ sample paths
\end{itemize}
such that they satisfy the equation $E(\drift,\diffusion)$.
If furthermore $X_0 = x$ we say that $X$ solves $E_x(\drift,\diffusion)$.
\end{definition}

To ensure existence (strong existence, meaning that we can find a solution on each probability space) and uniqueness (path uniqueness, meaning that having two solutions with the same initial conditions, they are indistinguishable)of solutions we need to require conditions on $\drift, \diffusion$. There are two main results:
\begin{itemize}
\item \autocite[Thm 8.3]{le-gall}: The two coefficients are Lipschitz (eventually only locally) in space and continuous:
\begin{equation*}
\left\{
\begin{aligned}
\forall t .( |\drift(t, x)-\drift(t,y)| &< L_\drift |x-y| &\text{ and } |\diffusion(t, x)-\diffusion(t,y)| &< L_\diffusion |x-y|) \\
\drift &\in \Cspc^0 &\text{ and }\diffusion &\in \Cspc^0
\end{aligned}
\right.
\end{equation*}
\item \autocite[Thm 9.4]{baldi}: The two coefficients are Lipschitz in space (as before) and have sublinear growth in space:
\begin{equation*}
\left\{
\begin{aligned}
\forall t .(|\drift(t, x)-\drift(t,y)| &< L_\drift |x-y| &\text{ and } |\diffusion(t, x)-\diffusion(t,y)| &< L_\diffusion |x-y|) \\
\forall t |\drift(t, x)| &< M_b(1 + |x|) &\text{ and } |\diffusion(t, x)| &< M_\diffusion (1 + |x|)
\end{aligned}
\right.
\end{equation*}\label{ch3:cond:E-and-U-lipsch+subgrowth}
\end{itemize}

\section{Reducing our system to the SDE framework, heuristically}

The formalization of the procedure illustrated here is treated in the appendix at \ref{app:formal-def-solution}.

Our system \ref{ch2:model:deterministic-parenting} is not a proper SDE because of technical oddities:
\begin{itemize}
\item An hypothetical vector $\modsol_t := (\amba_t^1,\ldots, \amba_t^N, \camp_t^{p,g} \text{ already spawned at time }t)$ would have a dimension changing (growing, in particular) as the time flows.
And it grows up to infinity.
\item Not only it would be changing with the time, but it would be a \emph{stochastic} dimension.
\item If we were to write the equations for $\modsol$ in a similar way to the one in $\ref{ch3:eq:general-sde}$ we would have a drift term $\drift = \drift(t, \modsol_t, (\hmactive_t^p)_p)$ depending also on the Poisson-distributed processes $(\hmactive_t^p)_p$, making it "too much stochastical" to be treated as in \ref{ch3:eq:general-sde}.
\end{itemize}

All these oddities originate from the fact that our system depends on $N$ sequences $(\btime^{p,g})_{g}$ of birth times.
We can notice that such sequences do not depend on any other process involved in the model (a part from $(\hmactive_t^p)_t$, to which they are completely equivalent)
Then heuristically we can imagine to fix these sequences, \ie choose a realization $(\tau^{p,g})_{g}$ of $(\btime^{p,g})_{g}$, and operate then as if they are deterministic.
Notice that choosing $(\tau^{p,g})_{g}$ is equivalent to take a sample path $(\mu_t^p)_{p,t}$ of $(\hmactive_t^p)_{t,p}$, with relations between $\tau^{p,g}$ and $\mu_t^p$ as the one intercurring between the respective processes. 
Then we have the system:

\begin{equation}
\left\{
\begin{aligned}
\amba_t^p 
&= \amba_0^p 
+ \int_0^t 
	\overbrace{
		\frac{1}{N} \sum _{p = 1}^{N} \sum_{g = 1}^{\mu_s^p} \attraction(\amba_s^p - \camp_s^{p,g})
	}^{=:\drift_p(s, \modsol_s, \mu_s^p)=:\drift_p(s, \modsol_s)} 
d s
+ 
\overbrace{\sigma^A}^{=:\diffusion_p(s,\modsol_s)} B_t^p
&
\\
\camp_t^{p,g} 
&\qquad\text{does not exist} 
&if \quad t < \tau^{p,g}
\\
\camp_t^{p,g}
&= \amba_{\tau^{p,g}}^p + \underbrace{\sigma_C}_{=:\diffusion_{p,g}(s,\modsol_s)} \quasibm_t^{p,g}
&if \quad t \geq \tau^{p,g}
\end{aligned}
\right.
\end{equation}

where the functions and variable involved are:

\begin{equation}
\left\{
\begin{aligned}
&\drift_p(t,x) = \frac{1}{N} \sum _{p = 1}^{N} \sum_{g = 1}^{\mu_s^p} \attraction(x^p - x^{p,g})
	&& 1\leq p \leq N
\\
&\drift_{p,g}(t,x) = 0
	&& \exists (p,g)
\\
&\diffusion_p(t,x) = \sigma_A
	&& 1\leq p \leq N
\\
&\diffusion_{p,g}(t,x) = \sigma_C
	&& \text{if } \exists (p,g)
\\
&x = (x^1, \ldots, x^N, x^{1,1}, x^{1,2}, \ldots, x^{p,1},\ldots, x^{p,g},x^{p,g+1},\ldots, x^{N,1}, \ldots)
	&& \text{if } \exists (p,g)
\end{aligned}
\right.
\end{equation}\label{ch3:def-determ-parent-drift-&-diffusion}

From here we can treat the system piecewise in time and obtain proper SDEs, with finite deterministic dimensions, as the one in \ref{ch3:eq:general-sde-integral-form} (the quasi-Brownian motions $Q^{p,g}$ will transform appropriately).
In fact we can order the birth times chronologically as

\begin{equation}
\tau_0:=0 < \tau_1 < \tau_2 < \ldots < \tau_m < \ldots
\end{equation}

(the correspondent sample path of $(\btime_m)_m$) and define the intervals

\begin{equation}
I_0 := [0,\tau_1], I_1:=[\tau_1,\tau_2], \ldots , I_m:= [\tau_m,\tau_{m+1}],\ldots
\end{equation}

On each interval $I_m = [\tau_m,\tau_{m+1}]$ the system can be described as this:

\begin{equation}
\left\{
\begin{aligned}
t &\in I_m = [\tau_m,\tau_{m+1}]
&
\\
\amba_t^p 
&= \amba_{\tau_m}^p 
+ \int_{\tau_m}^t 
	\frac{1}{N} \sum _{p = 1}^{N} \sum_{g = 1}^{\mu_s^p} \attraction(\amba_s^p - \camp_s^{p,g})
d s + \sigma^A B_t^p
&
\\
\camp_t^{p,g} 
&= \camp_{\tau^{p,g}}^{p,g} + \sigma_C \quasibm_t^{p,g}
&\text{if} \quad \tau^{p,g} < \tau_m
\\
\camp_t^{p,g}
&= \amba_{\tau^{p,g}}^p + \sigma_C \quasibm_t^{p,g}
&\text{when} \quad \tau^{p,g} = \tau_m
\end{aligned}
\right.
\end{equation}\label{ch3:det-parent-piece}

We observe that the point $\tau_{m+1}$ has measure $0$ in the Lebesgue measure $\lebsge$, and thus has measure $0$ for each probability measure which is absolutely continous under $\lebsge$.
This implies that we can ignore the fact that the particle $\camp_{m+1}$ is spawning at such time and still describe the system reliably in $I_m$, where the number of particles is temporarily constant, stabilising the dimension.

The only passage that we need in order to describe \ref{ch3:det-parent-piece} as a proper SDE is that of transforming the quasi-Brownian motions $\quasibm^{p,g}$ into proper Brownian motions.
This is easily done by shifting the system in time from $I_m$ to $I_m-\tau_m=[0, \tau_{m+1}-\tau_m]$, obtaining:

\begin{equation}
\left\{
\begin{aligned}
t &\in I_m - \tau_m = [0, \tau_{m+1}-\tau_m]
&
\\
\amba_t^p 
&= \amba_{\tau_m}^p
	+ \int_0^t 
		\overbrace{
			\frac{1}{N} \sum _{p = 1}^{N} \sum_{g = 1}^{\mu_s^p} \attraction(\amba_s^p - \camp_s^{p,g})
		}^{\drift_p}
	d s
	+ 
	\overbrace{\sigma^A}^{\diffusion_p} 
	\overbrace{(B_{t + \tau_m}^p - B_{\tau_m}^p)}
		^{\text{Brownian motion in }t}
&
\\
\camp_t^{p,g} 
&= \camp_{\tau^{p,g}}^{p,g} + \sigma_C \quasibm_{\tau_m + t}^{p,g} = \camp_{\tau^{p,g}}^{p,g} + \sigma_C B_t^{p,g}
&\text{if} \quad \tau^{p,g} < \tau_m
\\
\camp_t^{p,g}
&= \amba_{\tau^{p,g}}^p + \sigma_C \quasibm_{\tau_m + t}^{p,g}
	= \amba_{\tau^{p,g}}^p + \underbrace{\sigma_C}_{\diffusion_{p,g}} B_t^{p,g}
&\text{when} \quad \tau^{p,g} = \tau_m
\end{aligned}
\right.
\end{equation}\label{ch3:basic-SDE-piece}

and if this system has a solution ${}^m Y_t$ on $I_m - \tau_m$, then of course ${}^m X_t := {}^m Y_{t-\tau_m}$ solves \ref{ch3:det-parent-piece} on $I_m$.

\section{Ensure the existence of solutions?}

Now that we have reduced our model to a piecewise defined SDE, we can find individual solutions to each of them (each is defined on its interval $I_m$) and glue all of them together.
The glued solution will have, by definition, continuous sample paths for each particle from the moment of its existence on.
Even more, since each spawned particle $C^{p,g}$ is born on the position of its parent amoeba $A^p$, we will have a bifurcation there (\ie there are no jumps in the distance between particles at the moment of spawning).

Let us prove that the system \ref{ch3:basic-SDE-piece} has a solution.
On such system we have that the diffusion is a constant (and thus will not cause big problems, with the conditions we need) but that the drift is discontinous, precisely has potentially jumps at each time $\tau_m$:

\begin{equation}
\drift_p(t, x) := \frac{1}{N} \sum _{p = 1}^{N} \sum_{g = 1}^{\mu_s^p} \attraction(x^p - x^{p,g})
\end{equation}

then we must use the result \ref{ch3:cond:E-and-U-lipsch+subgrowth}, \ie prove that it is, at least locally, Lipschitz continuous, and that has sublinear growth.


[[DEFINE EXPLICITLY THE DIMENSION OF THE SYSTEM ON EACH TIME INTERVAL, SO THAT YOU CNA USE THIS IN THE FUTURE]]

\begin{lemma}

\end{lemma}



